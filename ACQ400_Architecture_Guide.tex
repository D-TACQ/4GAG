\documentclass[]{article}
\usepackage{import}

% Compile me with xelatex. It has proper support for UTF-8, Truetype and OpenType fonts out the box

% Define your document title here
\newcommand{\mytitle}{ACQ400 Architecture Guide}
\newcommand{\dtacqDOCREGNumber}{1305-9999-001} % D-TACQ Document Register Number
\newcommand{\dtacqORDERNumber}{Developer} % DOC Number
\newcommand{\vhCurrentVersion}{0.1}

\import{../../}{dtacq_doc_style} % Import general document options, custom colours, etc.
\import{../../}{dtacq_code_styles} % Import code styles

%\usepackage{booktabs}
\usepackage{underscore}   % breaks FAT if in dtacq_doc_style!

% Override standard font
\setmainfont{Liberation Sans}
% Override Left Footer to include DOC number
\lfoot{DOC-\dtacqDOCREGNumber\ - Rev \vhCurrentVersion}


\begin{document}

\title{ACQ400 Architecture Guide}
\author{Peter Milne \\ Contact: \href{mailto:peter.milne@d-tacq.co.uk}{peter.milne@d-tacq.co.uk} }

\thispagestyle{empty} % This suppresses the page number on titlepage
\begin{center}\huge{\textbf{\mytitle}}\end{center}
\vspace{2.5cm}
\begin{center}
	\includegraphics{../../dtacq_logo_new}
\end{center}
\begin{center}\textcolor{dtacqblue}{\large{\textit{High Performance Simultaneous Data Acquisition}}}\end{center}
\vspace{5cm}
%\begin{center}\large{\textcopyright \space D-TACQ Solutions Ltd. Confidential Product Document\\Only to be viewed by D-TACQ and authorised customers}\end{center}
\begin{center}\large{DOC \#XXXXXX-YY}\end{center}

%\begin{abstract}
%The abstract text goes here.
%\end{abstract}

\begin{versionhistory}
	\vhEntry{0.1}{2022-04-29}{PM}{Created}
	\vhEntry{0.2}{2022-02-17}{PM}{update to D-TACQ Style}
	\vhEntry{0.3}{2022-02-17}{PM}{appendix FPGA personality description}
\end{versionhistory}
\setcounter{table}{0} % Zeroes the table counter so the VerHist table isn't counted

\pagebreak

\tableofcontents


\section{Introduction}

\subsection{ACQ400 Architecture}\label{sec:daq-appliances}

The ACQ400 Architecture is a firmware framework used by all D-TACQ “4G” DAQ Appliances, based on the ZYNQ-7000 platform, including
\begin{enumerate}
	\item Compact DAQ Appliances: \href{https://www.d-tacq.com/acq1001Q.shtml}{ACQ1001}, \href{https://www.d-tacq.com/acq1002R.shtml}{ACQ1002}, \href{https://www.d-tacq.com/acq1014.shtml}{ACQ1014}
	\item Mainframe DAQ Appliances: \href{https://www.d-tacq.com/acq2106.shtml}{ACQ2106},  \href{https://www.d-tacq.com/acq2206.shtml}{ACQ2206}
	\item MTCA:  \href{https://www.d-tacq.com/mtca.shtml}{KMCU-Z30}, \href{https://innovation.desy.de/technologies/microtca/boards/damc_fmc1z7io/index_eng.html}{Z7IO}
	\item Custom: \href{https://www.d-tacq.com/news48.shtml}{CPSC2}
\end{enumerate}

This document describes the structure of the Embedded Software ESW in 3 levels of detail
\begin{enumerate}
    \item Overview - a brief description
    \item User/SystemIntegrator - a detailed at the boot process and structure of the elements
    \item Developer - describes how to build the system from scratch.
\end{enumerate}    

\subsection{Intended Audience}
\begin{enumerate}
    \item Users
    \item Developers
\end{enumerate}  

\subsection{Scope}
\begin{enumerate}
    \item ACQ400 Embedded Software
    \item excludes FPGA development.
\end{enumerate} 

\subsection{Glossary}
\begin{enumerate}
    \item \glossy{ACQ400} Embedded Software, excludes FPGA development.
    \item \glossy{DAQ Appliance}: a freestanding networked data acquisition system
    \item \glossy{ZYNQ 7000} (Z7, Z7000)  : Xilinx(R) system on chip architecture.
    \item \glossy{PS} : Processor System in \glossy{Z7000}
    \item \glossy{PL} : Programmable Logic (FPGA part) of \glossy{Z7000}
    \item \glossy{A9} : Dual Core ARM A9 processor found in  \glossy{Z7000}, the processor in \glossy{PS}
    \item \glossy{ESW} : Embedded SoftWare : software to run on the \glossy{A9}
    \item \glossy{Gateware}: programmable logic for the \glossy{PL}. This is delivered as an 
    \item \glossy{FPGA Personality} file.
    \item \glossy{Firmware Release}: combination of all \glossy{ESW} and \glossy{Gateware} needed to run an \glossy{ACQ400} System..
    \item \glossy{FMC}: Vita 57 standard for expansion modules
    \item \glossy{ELF} : D-TACQ extension of FMC : Analog module uses a limited subset of FMC I/O, optional larger size module, to allow a larger analog payload while at the same time using less FPGA pin resource.
    \item \glossy{Carrier/Motherboard/MB}: Electronic circuit board with Z7000 SOC, DRAM, Networking and one or more FMC or ELF payload sites.
    \item \glossy{QSPI Flash}: FLASH memory in the Z7000 system, typically used for customization and robust bootloader storage.  
    \item \glossy{MGT}: Multi Gigabit Transceiver links - ACQ2106:4x6Gbps, ACQ2206:4x10Gbps.
\end{enumerate}

\subsection{References}
\begin{enumerate}
    \item \href{https://github.com/seanalsop/bolodsp-doc/releases}{4GUG}
\end{enumerate} 

\subsection{Notation}
\begin{enumerate}
    \item \pcmd{command}  : indicates name of a program (command)
    \item \pftxt{preformatted text} : literal input or output from terminal session.
    \item \glossy{Defined Term} : some term or acronym specific to this domain (perhaps referenced in the glossary)
\end{enumerate} 

\pagebreak
\section{Overview}

\subsection{Single Software Image, Multiple FPGA Personalities}
The \glossy{ACQ400} firmware system runs on about 10 carrier platforms~\ref{sec:daq-appliances}, supporting a wide range (>10) analog payload \href{https://www.d-tacq.com/modproducts_modules.shtml}{modules}.  Each hardware combination of modules and sites, is supported by a unique FPGA personality~\ref{sec:fpga-personalities}. A single software image may handle all platforms (excepting CPSC2, which has a forked image).

\subsection{Embedded Linux System}
The \glossy{ACQ400} system runs Linux. A single Linux image supports all variants. This is the standard Linux kernel, and a fairly conventional userland, with two exceptions:
\begin{enumerate}
    \item Kernel DMA subsystem is non-standard.
    \item The system root file system rootfs is a RAMDISK. 
\end{enumerate} 

The RAMDISK rootfs is chosen for 
\begin{enumerate}
    \item Robustness: the file system is volatile, and the entire system may be reset or powered off at any time during operation. 
    \item Realtime performance: the DAQ appliance is a real time system, and must not be subject to any unpredictable delay eg disk access, or swapping.
\end{enumerate} 
The \glossy{ZYNQ 7000} has strictly limited DRAM: 1GB. In addition, \glossy{ACQ400} systems, being aimed at large scale data collection, typically assign at least half the DRAM to data buffers. So this is a restricted memory environment, and the size of the RAMDISK must therefore be kept in bounds.
The systems are typically fitted with an SD card, and the system boots from the SD card.

\begin{enumerate}
    \item The SD card is treated as a ROM. There’s no writeable access to the SD card at runtime on a production system.
    \item However, the SD card is writeable, this is convenient for system maintenance and for firmware upgrade.  After deployment, there should be no files opened for write while the unit is deployed, so it’s still safe to cut the power at any time.
\end{enumerate} 
The RAMDISK isn’t ideal for a limited memory system, and a pure ramdisk system (such as D-TACQ ACQ200) has a very limited userland. \glossy{ACQ400} has a rich user-land, based on Buildroot. Each \glossy{ACQ400} system also carries a large inventory (can be 100+) of FPGA personalities. \glossy{ACQ400} handles this by mounting additional read-only file systems from file images on the SD card. The OS then has access to a wide range of software images, loaded on demand. In a typical system, most of this functionality is swapped out at runtime, preserving DRAM.
Most systems are SD based, however two types of netboot systems are also supported:
\begin{enumerate}
    \item Pure TFTP : the entire system boots from a RAMDISK image loaded via TFTP. This has a limited userland, and it does depend on a TFTP server for boot, but once running is self-standing and extremely robust. CPSC2 is the main example of this method.
    \item NFS : the pure TFTP system boots a limited userland, then brings in the rest of the userland as an NFS mount. This allows high functionality with limited stored state, but it does totally depend on a reliable NFS server during run time.
\end{enumerate} 

\subsection{Self Identifying System}
\begin{enumerate}
    \item On boot, \glossy{ESW} will identify which motherboard it is running on.
    \item Then it will use SMB (i2c) to identify the current payload 
        	ie which analog modules are present in the module sites
    \item The system will then select the first available FPGA personality, and load the FPGA
    \item The system then proceeds to load kernel modules, instantiate device drivers and all services needed to build the system.
\end{enumerate}
=> The same \glossy{ESW} image can run any supported personality, based on runtime identification.
Z7IO: currently, the \glossy{ESW} is not able to complete system indentification before loading the FPGA, so this is selected at boot time.

\subsection{Device Driver and Memory Map}
The system hardware bus layout and memory map is defined by the devicetree.

\subsubsection{I2C Devices}
I2C/SMB is used extensively, not least for module identification and monitoring as well as slow control (eg gain setting).
The i2C bus layout is defined 
\begin{enumerate}
    \item statically in the main boot time devicetree  
    \item dynamically during system enumeration, either by
    \begin{enumerate}
	\item loading custom kernel modules that define the devices directly (old method).
	\item loading devicetree overlays (new method).
    \end{enumerate}	
\end{enumerate}

Example i2c bus layout (ACQ2206):

\begin{lstlisting}[language=bash,style=bashstyle,frame=single]
acq2206_001> ls  /sys/bus/i2c/devices/ | grep -v ^i2c | sort -n
i2c-0		# root
i2c-1		# Site0 : motherboard
i2c-2		# Site1
i2c-3		# Site2
i2c-4		# Site3
i2c-5		# Site4
i2c-6		# Site5
i2c-7		# Site6
i2c-8		# MGT482 root
i2c-9		# MGT482 expansion
i2c-10
i2c-11
i2c-12
i2c-13
i2c-14
i2c-15
i2c-16

\end{lstlisting}

\subsubsection{AXI Bus Devices for control}
The FPGA instantiates an AXI bus memory mapping, and this is preconfigured into logical device areas, notably by module site

Example from ACQ2206, 2 sites present.
\begin{lstlisting}[language=bash,style=bashstyle,frame=single]
cat /proc/iomem
40000000-4000ffff : /amba/acq2006sc@40000000	# Site0
40010000-4001ffff : /amba/acq400fmc@40010000	# Site1
40020000-4002ffff : /amba/acq400fmc@40020000	# Site2
400c0000-400cffff : mgt400.B			# CommsB
400d0000-400dffff : mgt400.A			# CommsA
801f0000-801fffff : /amba/axidma@801f0000	# AXI DMA
\end{lstlisting}

\subsubsection{AXI Bus Devices for DMA}
The \glossy{ACQ400} system makes extensive use of DMA for moving data at high rates. The system supports the following DMA engines
\begin{enumerate}
    \item pl330 : a hard-core DMA in the \glossy{Z7000} (low rate, legacy apps)  
    \item AXI32 : 32 bit \glossy{AXI} DMA instantiated in \glossy{PS}, for lower rate systems, back-compatibility with pl330
    \item AXI64 : 64 bit \glossy{AXI} DMA instantiated in \glossy{PS}, for high rate systems (to 960MB/s)
    \item 2D AXI64 : 2x 64 bit \glossy{AXI} DMA instantiated in \glossy{PS}, for very high rate systems (to 1920MB/s)
    \item COMMS : DMA engines configured to control data transfer on \glossy{MGT}
\end{enumerate}

\subsubsection{FPGA Device Drivers}
\begin{enumerate}
	\item \pftxt{acq420fmc.ko} : main device driver, one driver works for all motherboards (Site 0) and all payloads (Site1-6).
	\item \pftxt{mgt400.ko}    : device driver for MGT48x, handles comms DMAC.
	\item \pftxt{acq400_dspfs} : instantiates a user-programmable register map for expansion DSP.
\end{enumerate}
The \pftxt{acq420fmc.ko} device driver functions are:
\begin{enumerate}
	\item map each memory region according to the devices that are fitted.
	\item identify each site payload and instantiate init, run, stop routines as appropriate, usually run from char device driver entries
	\item create a set of knobs for each payload (a sysfs control for each programmable element).
	\item realtime buffer management for data flow during capture/replay.
\end{enumerate}
The site knobs appear like this:
\begin{lstlisting}[language=bash,style=bashstyle,frame=single]
acq2206_001> ls -1 /dev/acq400.0.knobs/
AXI_DMA_len
..
data32
data_engine_0
data_engine_1
.
dist_bufferlen
dist_s1
distributor
..
\end{lstlisting}

\subsubsection{Site Services}
Site services are published as TCP socket servers on well known ports (Site0=4220, Site1=4221).
The site services are represented by a directory of virtual files; these can be
\begin{enumerate}
	\item soft links to device knobs (above)
	\item simple ascii data files
	\item links to shell scripts
\end{enumerate} 
The site server \pcmd{acq400_knobs} scans this directory and enables the site service.
eg
\begin{lstlisting}[language=bash,style=bashstyle,frame=single]
# dssb links to a sysfs knob, ssb is a small local file. transacq400_knobs clients are unaware..
acq2206_001> ls -l /etc/acq400/0/ | grep ssb 
lrwxrwxrwx    1 root     root            59 Jan  1 00:00 dssb -> /sys/module/acq420fmc/parameters/xo_distributor_sample_size
-rw-r--r--    1 root     root             3 Jan  1 00:00 ssb
acq2206_001> ls -l /etc/acq400/0/ | grep data | cut -c58-

# data* are all links to sysfs.

data32 -> /dev/acq400.0.knobs/data32
data_engine_0 -> /dev/acq400.0.knobs/data_engine_0
data_engine_1 -> /dev/acq400.0.knobs/data_engine_1
data_engine_2 -> /dev/acq400.0.knobs/data_engine_2
data_engine_3 -> /dev/acq400.0.knobs/data_engine_3

# *FREQ are links to EPICS CA values, transparent to client0
acq2206_001> ls -l /etc/acq400/0/ | grep FREQ | cut -c58-
SIG:CLK_EXT:FREQ -> /usr/local/bin/caget_acq2206_001:0
SIG:CLK_MB:FREQ -> /usr/local/bin/caget_acq2206_001:0
SIG:CLK_S1:FREQ -> /usr/local/bin/caget_acq2206_001:0
SIG:CLK_S2:FREQ -> /usr/local/bin/caget_acq2206_001:0
SIG:CLK_S3:FREQ -> /usr/local/bin/caget_acq2206_001:0
SIG:CLK_S4:FREQ -> /usr/local/bin/caget_acq2206_001:0
..
\end{lstlisting}


\subsection{ESW Components}

\subsubsection{Bootloader: u-boot}
The system bootloader. u-boot will read local customization from the u-boot environment, then boot the system.

\subsubsection{uImage: kernel image}
The kernel image contains the Linux kernel executable code.

\subsubsection{initrd : initial ramdisk image}
The initial filesystem image, with limited functionality, this provides the first userland after boot.

\subsubsection{devicetree}
This is a data structure that defines the architecture of the particular motherboard, and any customisations.
\begin{enumerate}
    \item eg ACQ1001 or ACQ2106: \pftxt{dtb.d/acq1001.dtb}, \pftxt{dtb.d/acq2106.dtb}
    \item eg ACQ2106 (no sfp) or ACQ2106sfp (with sfp)., \pftxt{dtb/acq2106.dtb}, \pftxt{dtb.d/acq2106sfp.dtb}
\end{enumerate}
The boot-time devicetree may be augmented later in the boot-sequence by "devicetree-overlays", that define additional devices once more is known about the system.

\subsubsection{rootfs.tgz}
This is the main userland. This is created using Buildroot. This is expanded onto the SD card at install time, then mounted onto the ramdisk at run time, with the goal that most of the userland will NOT be resident in memory at runtime.

\subsubsection{Filesystem overlays}
These are read-only file system images mounted onto the ramdisk, and most of these images are NOT resident in memory at runtime. The most prominent overlays are:
\begin{enumerate}
    \item FPGA stock: a “stock” of all known FPGA personalities; one of these files may be selected to load the FPGA on boot.
    \item ko stock : a stock of kernel objects (device drivers), and appropriate modules will be loaded on boot.
    \item EPICS7 : a full EPICS7 implementation is quite large, so it is shipped as an overlay rather than loading everything to RAMDISK.
\end{enumerate}

\subsubsection{Packages}
\glossy{ACQ400} uses a sequenced set of packages to configure the system; packages are tarballs that are expanded onto the ramdisk in sequence, together with initialisation files that are used to instantiate the entire system. Packages may reference overlays to save RAM, and they may load devicetree-overlays to instantiate new devices.
A package is a tarball:
\pftxt{SEQ-NAME-YYMMDD.tgz}
\begin{enumerate}
	\item \pftxt{SEQ}	:  SEQuence number 01..99 : defines the order that the package is installed
	\item \pftxt{NAME}	:  NAME must be unique
	\item \pftxt{YYMMDD}:   Date code for versioning
\end{enumerate}
The tarball may contain:
\begin{enumerate}
	\item \pftxt{files/to/install}   : files are copied to the RAMDISK
	\item \pftxt{/usr/local/init/\pcmd{NAME.init}} : after the copy, this intialization file is run
\end{enumerate}

\subsection{SD Boot or QSPI boot}
Z7 systems typically have a jumper setting to boot from SD (certainly, for all newly assembled units), or from QSPI (soldered flash). On D-TACQ motherboards, initial boot is from SD, then the QSPI is programmed, and the jumper set to QSPI boot.
The QSPI contains:
\begin{enumerate}
    \item (minimum) : u-boot environment ie per-motherboard customization.
    \item (ideally) : the bootloader, then a motherboard can boot to a prompt even with no SD card present, and remote recovery is possible.
\end{enumerate}

\subsection{Boot Sequence}

\subsubsection{BOOT.BIN}
First to load is BOOT.BIN, a file created with Xilinx tools containing:
\begin{enumerate}
    \item (minimum) : FSBL: First Stage Bootloader, a fixed boot kernel, and a motherboard specific IO configuration.
    \item (ideally) : u-boot : the main bootloader.
\end{enumerate}
As per 2.5 XREF HOWTO?, in a production system this image is held in QSPI for robustness

\subsubsection{u-boot}
The u-boot environment includes a BOOTCMD macro that specifies:
\begin{enumerate}
    \item module ID (serial number, mac address) :: (ACQx0xx)
    \item devicetree
    \item kernel image
    \item initrd        
\end{enumerate}
BOOTCMD loads the three images, and boots uImage

\subsubsection{uImage}
This is a standard Linux ARM kernel boot up, the Linux system is configured, and all the devices in the devicetree are instantiated, most importantly:
\begin{enumerate}
    \item Ethernet eth0 (or eth1)
    \item All i2c proms in the self-identification system    
\end{enumerate}
Finally uImage sets up the initrd and hands over to userland.

\subsubsection{initrd}
The init system runs \pftxt{/etc/init.d/\pcmd{rcS}}, a highly customized script that will
\begin{enumerate}
    \item mount the rootfs file system
    \item perform any custom network configuration
    \item load packages in sequence.
\end{enumerate}

\subsubsection{packages}
This is the real DAQ Appliance application userland. 

\subsubsection{/mnt/local/rc.user}
Last to run is the file \pftxt{/mnt/local/\pcmd{rc.user}}, this contains final factory and/or user customisation. \pcmd{rc.user} can be configured to make a completely turnkey system.

\subsection{Package Sequence}


\subsubsection{03-acq400_common}
Basic setup common to all platforms.

\subsubsection{05-machine}
Machine-specific setup eg \pftxt{05-acq1001-yymmdd.tgz}.
In particular, this will enumerate the site payload modules, select an \glossy{FPGA Personality} and load the FPGA~\ref{sec:fpga-personalities}.

\subsubsection{10-acq420}
This is the main device driver and operating code of the \glossy{ACQ400} system.
Once the FPGA has been loaded, then the core device driver \pftxt{acq420fmc.ko} is loaded, and this instantiates a Linux device per site, bringing up all site services, and loading associated kernel modules, notably to instantiate DMA controllers. The package also provides \pcmd{acq400stream}, the main data-moving application.

\subsubsection{20-httpd}
Configures and starts the embedded webserver. The embedded webserver provides useful system observability and diagnostics.

\subsubsection{40-acq400ioc}
This package loads and configures the embedded EPICS IOC, responsible for running most of the “business logic” of the DAQ appliance, and providing a rich set of control and monitor points for external clients.

\subsubsection{Custom Packages}
Systems may be customized by installing “nn-custom_PACKAGE modules.
"install" copy the package to the active package directory: /mnt/packages

\subsection{SD File System Layout}
The SD card appears in the booted linux system as /mnt

\begin{table}[H]
\ttfamily
\begin{tabular}{|l|l|l|}
\hline
\textbf{Dir} & \textbf{File} & \textbf{Description} \\
\hline
/mnt/ko & & mountable file system images eg  \\ 
\hline 
& fpga-511-yymmdd.img & FPGA STOCK \\ 
\hline 
& packageko-4.14.0-acq400 & kernel object STOCK\\&-xilinx-yymmdd.img  \\ 
\hline 
/mnt/packages & & live packages are found here \\ 
\hline 
/mnt/packages.opt & & optional (inactive) packages \\ 
\hline 
/mnt/bin & & scripts for firmware upgrade \\ 
\hline 
/mnt/boot.d & & BOOT.bin instances for supported carriers. \\ 
\hline 
/mnt/dtb.d & & stock of devicetree instances \\ 
\hline 
/mnt/fpga.d & & FPGA stock mounted here \\ 
\hline 
/mnt/local   & & user/system customisation.. \\
\hline 
/mnt/local/cal & & calibration files for current payload \\
\hline 
/mnt/local/sysconfig & &  system customization. \\
\hline 
\end{tabular} 
\end{table}


\subsection{SD card top-level files}
\begin{table}[H]
\ttfamily
\begin{tabular}{|l|l|}
\hline	
\textbf{File} & \textbf{Description} \\	
\hline 
BOOT.BIN & The specific bootloader for this carrier (for SD boot) \\ 
\hline 
uImage & The kernel image \\ 
\hline 
uramdisk.image.gz & The initial ramdisk initrd \\ 
\hline 
rootfs.ext2 & The root file system, to be mounted onto the ramdisk \\ 
\hline 
Optionally: & \\ 
\hline 
ACQ2106_TOP_04_*_9011_32B_UDP.bit.gz: & Promoted \glossy{FPGA Personality} **\\ 
\hline 
\end{tabular}
\end{table}

** This is a “PROMOTED” \glossy{FPGA Personality} - provided this image is compatible with the module payload, it will be loaded in preference to the first compatible image in the fpga stock. This is useful for:
\begin{enumerate}
    \item Development: load a test image in preference to the release image.
    \item Patch: when the \glossy{FPGA Personality} doesn’t exist in the release yet.
    \item Promoting: when the stock contains multiple compatible images, and we want to force a particular \glossy{FPGA Personality} to load eg: DEC10 filter.
\end{enumerate}

\pagebreak
\section{User/SystemIntegrator Guide}

\subsection{u-boot environment}

\subsubsection{acqXXX factory settings}

The u-boot environment of ACQ products are customised at the factory on assembly.
All unit ID, the serial number, mac addresses are held in u-boot environment.

\begin{lstlisting}[language=bash,style=bashstyle,frame=single]
acq2206_001> cat /tmp/u-boot_env 
# Kernel command line
bootargs="console=ttyPS0,115200 root=/dev/ram rw earlyprintk hostname=acq2206_001" 
# u-boot boot mode
bootcmd="run sdboot"
bootdelay="3"

# Device tree selection: has large effect on boot time customisation
devicetree_image="dtb.d/acq2206sfp.dtb"
# Audit trail for this customisation
dtenvid="root@rpi-008 R1010 Fri Jan 27 14:29:53 UTC 2023"
dtsn="CE4260001"						# serial number
eth1addr="00:21:54:24:00:01"					# ethernet mac addresses
eth2addr="00:21:54:34:00:01"
ethaddr="00:21:54:14:00:01"
fdt_high="0x20000000"
hn="hostname=acq2206_001"					# hostname
kernel_image="uImage"						# boot this kernel
modeboot="sdboot"
ramdisk_image="uramdisk.image.gz"				# with this ramdisk
\end{lstlisting}


\subsubsection{z7io}
The z7io u-boot environment from DESY must be altered to boot ACQ400.
z7io has NO unit-specific customization in the u-boot environment. Instead, the z7io u-boot reads the eth0 MAC address from a soldered PROM.

\begin{lstlisting}[language=bash,style=bashstyle,frame=single]
bootargs=earlyprintk console ttyPS0,115200 root=/dev/mmcblk0p2 rw rootwait
bootargs_acq400=earlyprintk console=ttyPS0,115200 root=/dev/ram rw rootwait
bootcmd=run sdboot
bootkernel=run loadkernel && run loaddtb && bootm ${kernel_load_address} - ${devicetree_load_address}
devicetree_acq400=dtb.d/z7io.dtb
devicetree_load_address=0x2000000
kernel_acq400=uImage
kernel_image=uImage
kernel_load_address=0x2080000
loaddtb=fatload mmc 0 ${devicetree_load_address} ${devicetree_image}
loadfpga=fatload mmc 0 ${bitstream_load_address} ${bitstream_image} && fpga ${bitstream_type} 0 ${bitstream_load_address} ${filesize}
loadkernel=fatload mmc 0 ${kernel_load_address} ${kernel_image}
machine_name=damc-fmc1z7io
mfr=DESY
ramdisk_acq400=uramdisk.image.gz
sdboot=setenv bootargs ${bootargs_acq400} && echo Copying Linux from SD to RAM... && mmcinfo && fatload mmc 0 0x3000000 ${kernel_acq400} && fatload mmc 0 0x2A00000 ${devicetree_acq400} && fatload mmc 0 0x2000000 ${ramdisk_acq400} && bootm 0x3000000 0x2000000 0x2A00000
uenvboot=run uenvcmd
uenvcmd=run loadfpga && run bootkernel
\end{lstlisting}

\subsection{kernel options}
Specified with \pftxt{\$\{bootargs\}} in the u-boot environment. No further customisation is required.

\pagebreak
\subsection{userland options}

Sysconfig is pre-configured according to application at the factory

\begin{lstlisting}[language=bash,style=bashstyle,frame=single]

acq2206_001> ls -1 /mnt/local
acq420_custom       # pre-customisation for device driver load
rc.user             # final stage boot customisation
wr_cal              # White Rabbit calibration
wrc.le.bin          # White Rabbit firmware

acq2206_001> ls -R1 /mnt/local/sysconfig/
acq400.sh           # sets major parameters like NBUF, BLEN
bos.sh              # long transient limits
epics.sh            # EPICS customisation
site-1-peers            # peers
transient.init          # early stage customisation
wr.sh               # White Rabbit variables

ls /mnt/local/cal
E4ssssssss.xml			# module calibration files.
\end{lstlisting}
EPICS customisation see \href{https://github.com/D-TACQ/acq400ioc/blob/epics7/README}{EPICS README}

\subsection{packages}
Optional packages can be enabled by \pcmd{mv} ing them from \pftxt{/mnt/packages.opt} to \pftxt{/mnt/packages}

\pagebreak
\section{Developer Guide}

\subsection{Build ESW}

\subsubsection{ESW}
The entire \glossy{ACQ400} \glossy{ESW} software suite is Open Source, and is presented by a single git top level archive with submodules:
\href{https://github.com/D-TACQ/ACQ400_ESW_TOP.git}{ACQ400_ESW_TOP}
 and may be built from scratch by following these 
\href{https://github.com/D-TACQ/ACQ400_ESW_TOP/blob/main/README.md}{build instructions}


\subsubsection{Making a Complete Firmware Release}
A complete bootable SD card image may be assembled as follows:
\begin{enumerate}
	\item Build \glossy{ESW}
	\item Include appropriate BOOT.BIN if needed
	\item Include appropriate FPGA filesystem image.
	\item Package and deploy the release
\end{enumerate}


\subsection{BOOT.BIN}

\subsubsection{z7io}
z7io BOOT.BIN is built from DESY sources.

\subsection{kernel and devicetrees}

\pagebreak

\section{APPENDIX}
	
	
\subsection{FPGA PERSONALITIES}\label{sec:fpga-personalities}

Let's try explain some personalities. Each Module has a unique "MTYPE", so that we tie the module in the site to the FPGA personality. 
\href{https://www.d-tacq.com/resources/Products_ACQ400_ModuleIDs.pdf}{ACQ400_ModuleIDs.pdf}

For example:
\begin{enumerate}
	\item ACQ435 has M=02
	\item ACQ465 has M=0A
	\item AO424  has M=41
	\item DIO482 has M=6B
\end{enumerate}

You can see the current stock of personalities on any \glossy{ACQ400} unit, in /mnt/fpga.d

\begin{lstlisting}[style=bashstyle,frame=single]
# 162 personalities in stock!
acq2206_001> find /mnt/fpga.d -type f -print | wc
162       162      8296
\end{lstlisting}

\begin{lstlisting}[style=bashstyle,frame=single]
# 85 of them for ACQ2106
acq2206_001> find /mnt/fpga.d -type f -print | grep ACQ2106 | wc
85        85      5103
\end{lstlisting}

\begin{lstlisting}[style=bashstyle,frame=single]
# what support for ACQ465 do we already have?
acq2206_001> find /mnt/fpga.d -type f -print | grep ACQ2106 | grep 0A
/mnt/fpga.d/ACQ2106_TOP_0A_0A_0A_0A_0A_0A_9011_32B.bit.gz
# format
/mnt/fpga.d/CARRIER_____S1_S2_S3_S4_S5_S6_COMMS_FEATURES
\end{lstlisting}

\begin{lstlisting}[style=bashstyle,frame=single]
# ZERO released personalities so far for ACQ2206
acq2206_001> find /mnt/fpga.d -type f -print | grep ACQ2206 | wc
0         0         0

# So how does this unit work?. We patched in an appropriate personality to load preferentially:

acq2206_001> cat /tmp/fpga_status
load.fpga loaded /mnt/ACQ2206_TOP_02_02_02_02_02_02_9011_32B.bit.gz
xiloader r1.01 (c) D-TACQ Solutions
eoh_location set 0
Xilinx Bitstream header.
built with tool version   : 48
generated from filename   : ACQ2206_TOP_02_02_02_02_02_02_9011_32B
part                      : 7z030ffg676
date                      : 2023/02/09
time                      : 17:45:05
bitstream data starts at  : 134
bitstream data size       : 5979916

ERROR: loaded FPGA image NOT in RELEASE at all

# this is compatible with the current payload: M=02 in S1, and the COMMS module M=90

CARRIER
SITE      MANUFACTURER           MODEL                                PART          SERIAL
0     D-TACQ Solutions      acq2206sfp                          acq2206sfp       CE4260001
---------------------------------------------------------------------------------------------
build detail: root@rpi-008 R1010 Fri Jan 27 14:29:53 UTC 2023
eth0 macaddr: 00:21:54:14:00:01    eth0  ipaddr: 10.12.197.98
eth1 macaddr: 00:21:54:24:00:01    eth1  ipaddr:
---------------------------------------------------------------------------------------------

MODULES
SITE      MANUFACTURER           MODEL                                PART          SERIAL
1     D-TACQ Solutions       ACQ435ELF         ACQ435ELF-32FF-5V N=32 M=02       E43510193
C     D-TACQ Solutions          MGT483                MGT483-SFP4 N=3 M=90       AM4831001

# If a site has no module, then it's a wildcard match.
\end{lstlisting}
\pagebreak
\begin{lstlisting}[style=bashstyle,frame=single]
# eg what support for ACQ435 do we already have?
acq2206_001> find /mnt/fpga.d -type f -print | grep ACQ2106 | grep 02
/mnt/fpga.d/ACQ2106_TOP_02_02_02_02_02_02_9011.bit.gz
/mnt/fpga.d/ACQ2106_TOP_02_02_02_02_02_02_9011_32B-D37.bit.gz
/mnt/fpga.d/ACQ2106_TOP_02_02_02_02_02_02_9011_32B.bit.gz
/mnt/fpga.d/ACQ2106_TOP_02_02_02_02_02_02_9011_32B_WR-D37.bit.gz
/mnt/fpga.d/ACQ2106_TOP_02_02_02_02_02_02_9011_32B_WR.bit.gz
/mnt/fpga.d/ACQ2106_TOP_02_02_02_02_41_61_9011.bit.gz
/mnt/fpga.d/ACQ2106_TOP_02_ff_02_ff_02_ff_9011_32B_WR_UDP-FFC.bit.gz
/mnt/fpga.d/ACQ2106_TOP_02_ff_02_ff_02_ff_9011_WR_32B-FFC.bit.gz
\end{lstlisting}

\begin{lstlisting}[style=bashstyle,frame=single]
# and for AO424 and DIO482 ?  ... quite a lot of options, usually with 09 (ACQ423), common in PCS applications.

acq2206_001> find /mnt/fpga.d -type f -print | grep ACQ2106 | grep 41 | grep 6B
/mnt/fpga.d/ACQ2106_TOP_09_09_09_09_41_6B_9011_32B_WR-PG.bit.gz
/mnt/fpga.d/ACQ2106_TOP_09_09_09_09_41_6B_9011_32B_WR.bit.gz
/mnt/fpga.d/ACQ2106_TOP_09_09_09_09_41_6B_9011_32B_WR_UDP.bit.gz
/mnt/fpga.d/ACQ2106_TOP_09_09_41_7B_7B_6B_9011_32B_WR_UDP.bit.gz
/mnt/fpga.d/ACQ2106_TOP_41_41_41_41_61_6B_9011-PWM.bit.gz
/mnt/fpga.d/ACQ2106_TOP_41_41_41_41_61_6B_9011-PWMFAST.bit.gz
/mnt/fpga.d/ACQ2106_TOP_41_41_41_41_61_6B_9011-PWMPROG.bit.gz
/mnt/fpga.d/ACQ2106_TOP_41_41_41_41_61_6B_9011_32B_WR-PWMPROG.bit.gz
/mnt/fpga.d/ACQ2106_TOP_41_41_41_41_61_6B_9011_WR-PG.bit.gz


Let's assume we pick two personalities:
1. ACQ465 in sites 1,2,3,4 AO424 in 5, DIO482 in 6
2. ACQ435 in sites 1,2,3,4 AO424 in 5, DIO482 in 6

The files would be:
/mnt/fpga.d/ACQ2206_TOP_02_02_02_02_41_6B_COMMS_EXTRA
and
/mnt/fpga.d/ACQ2206_TOP_0A_0A_0A_0A_41_6B_COMMS_EXTRA
\end{lstlisting}

%\usepackage{docmute}
%\include{module_ids.tex}

\end{document}
